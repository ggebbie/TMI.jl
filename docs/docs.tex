
\documentclass[agums]{aguplus}  % use this variant for AGU manuscripts
%\documentclass[agums]{aguplus_ams}  % use this variant for to get parentheses for 
                                    % American Meteorlogical Society publications
%\usepackage{psfig,times}
\usepackage{wasysym}
\usepackage{graphicx}
\usepackage{pslatex}
\usepackage{subfigure}
% \usepackage{setspace}
 \usepackage[dvips]{epsfig}
 \setlength{\oddsidemargin}{0.1cm}
 \setlength{\evensidemargin}{0.1cm}
 \setlength{\textwidth}{16cm}
 \setlength{\textheight}{22.1cm}
 \setlength{\topmargin}{.0cm} 
%\usepackage{amsmath}
\usepackage[latin1]{inputenc}
\def\degree{\hbox{$^\circ$}}
\usepackage[T1]{fontenc}
\def\scalefig#1{\epsfxsize #1\textwidth}

\makeatletter
%\def\@sluginfo{{\vspace{1in}in preparation, \today}}
\renewcommand\revtex@pageid{}
\makeatother


%\renewcommand\NAT@open{(} \renewcommand\NAT@close{)}
%\makeatother
\sectionnumbers
%\printfigures
%\tighten  %uncomment this line for single-spaced text
\lefthead{Gebbie}

\righthead{Auxiliary Text}
\received{}
\revised{}
\accepted{}
\journalid{}{}
\articleid{}{}
\paperid{}
\cpright{}{}
\ccc{}
%\printfigures


\begin{document}

% \title{Auxiliary Text and Figures: How is the ocean filled?}

% \author{Geoffrey Gebbie$^{1,2,3}$ and Peter Huybers$^1$}

\title{}
% \affil{$^1$ Harvard University, Cambridge, MA, $^2$ Massachusetts Institute of Technology, Cambridge, MA, $^3$ Now at Woods Hole Oceanographic Institution, Woods Hole, MA} 
       

% %\author{In press, {\it Geophys. Geochem. Geosyst.}, \today}  % place holder for drafts
% \authoraddr{Geoffrey Gebbie, Woods Hole Oceanographic Institution, MS \# 29, 266 Woods Hole Rd., Woods Hole, MA, 02543, USA.}

% %% agu abstracts. GRL.
% %   *  Set as a single paragraph
% %    * Limit to 250 words for journals
% %    * Do not include references in the abstract

%\begin{abstract}

% \end{abstract}

% \begin{flushright}
% \textbf{Auxiliary Information}
% \end{flushright}


% \vspace{1in}

% \begin{center}
% \large{\textbf{How is the ocean filled?}}\\
% \vspace{.25in}

% G. Gebbie \verb+(ggebbie@whoi.edu)+ and P. Huybers.

% \end{center}


% \vspace{.5in}

% \begin{itemize}

% \item Auxiliary Notes and Figures

% \begin{itemize}

% \item       The steady-state circulation

% \item	Estimating tracer distributions

% \item	Diagnosing water-mass pathways

% \item Adjoint method for the surface origination map

% \item The distribution of water masses and its sensitivity to spatial resolution

% \end{itemize}

% \item References

% \item Figures

% \end{itemize}

%\Large{\textbf{Auxiliary Text and Figures}} \\

%\normalsize

\begin{itemize}
 \item[] {\bf Auxiliary Material Submission for manuscript} \\
  ``How much did Glacial North Atlantic Water shoal?'' \\ \\
Geoffrey Gebbie \\
Physical Oceanography Department, Woods
Hole Oceanographic Institution \\ \\

\item[]    Contents 

\item[1.] Detailed inverse problem formulation

\item[2.] Model-Data Misfits

\item[3.] Water-mass analysis

\item[4.] References

\item[5.] Figures

\item[5b.] Western Atlantic Property Atlas

\item[5c.] Eastern Atlantic Property Atlas


%\item[3.] Brazil Margin analysis


%\item Gebbie, G., and P. Huybers, How is the ocean filled?, 2011. 

 \end{itemize}

% Introduction 


% 1. 2010jd014771-txts01.doc: Text S1: 
% (1) A summary of the infrared spectra for Figures 1 and 5 in the overall range that were measured, along with the theoretical ones in the 0-2500 cm-1 wavenumber range. 
% (2) The final structures used in theoretical calculation in Cartesian coordinate format for all compound from Tables 3 and 5. Calculation were run in Gaussian03 with the B3LYP/6-31G** level of theory. X, Y and Z represent the coordinates in A (deg).

\newpage

\section{Detailed inverse problem formulation}
\label{sec:proxy}

The method is detailed through a set of constraints on the problem
unknowns: the water-mass proportions, ${\bf m}$, all relevant tracer
distributions on an underlying grid, ${\bf c}_k$, and any
remineralization source for nonconservative tracers, ${\bf q}$.

%\subsection{Constraints}

Tracer observations are imposed without loss of generality as an
equation: $\sum^K_{k=1} \{ {\cal E}_k[ {\bf c}_k] \} = {\bf y} + {\bf
  n}$, where there are $K$ modeled tracers, ${\cal E}_k$ is a
potentially nonlinear mapping of the gridded fields onto the observational
location of observational type and performs the ``proxy'' step of
relating the modeled tracers to observed quantities, ${\bf y}$ is the
list of many disparate observations, and ${\bf n}$ is the noise in the
observations. This form takes into account the fact that
paleo-observations may depend upon multiple seawater tracers (e.g.,
$\delta^{18}$O$_c$ requires temperature and seawater $\delta^{18}$O),
and that this relationship may be nonlinear (e.g., $\delta^{18}$O$_c$
requires in-situ rather than potential temperature, and Cd$_w$ is
nonlinearly related to phosphate). If the tracer observations are
restricted to those that are explicitly modeled (e.g.,
$\delta^{18}$O$_w$) or to climatologies where a tracer value is
available at all gridded locations (e.g., WOCE salinity and
nutrients), the nonlinear function ${\cal E}_k$ is reduced to a
matrix. % or even drops out of the problem entirely.

The tracers are constrained by non-observational information, as well.
These constraints include gravitational stability, non-negativity
constraints expressed in terms of barrier functions, and
$\delta^{13}$C$_{as}$ being within the range that has been found in
modern-day observations and models ($\pm 1\permil$).  To keep
the solution within reasonable bounds and also improve the efficiency
of the solution method, the surface tracer deviation is defined
relative to a first guess, $\Gamma {\bf c}_{bk} = {\bf c}_{0k} + {\bf
  n}_{bk}$, where $\Gamma$ picks out the surface values from the
global distribution, ${\bf c}_{bk}$ is the surface concentration,
${\bf c}_{0k}$ is the first guess, and ${\bf n}_{bk}$ is surface
deviation away from the first guess for tracer $k$. Equations for
gravitational stability, non-negativity, and the range of
$\delta^{13}$C$_{as}$ are symbolically lumped into one set of
nonlinear equations for the derivation here: ${\cal F}_c[{\bf c}_k] =
{\bf n}_c$, where ${\bf n}_c$ is the degree to which these equations
are inexact.

No direct observations of the interior tracer sources, ${\bf q}$, or
the water-mass pathway vector, ${\bf m}$, are available, so these
unknowns are simply kept within a reasonable range of a first-guess
where, for example, ${\cal F}_q[{\bf q},{\bf q}_0]=\mbox{log}({\bf q})
- \mbox{log}({\bf q}_0)={\bf n}_q$, is the deviation from the first
guess local sources, ${\bf q}_0$. The difference of logarithms assumes
and enforces the source to be positive, and for the order of magnitude
of the deviation to be penalized. The pathway parameters, ${\bf m}$,
are penalized using an similar nonlinear function, ${\cal F}_m$.

\subsection{Steady state constraint}

Equation~(1) in the main text is put into the form of a steady-state
constraint by putting all terms on the left hand side:
$f_{ik} = \sum^N_{j=1} m_{ij}~ c_{jk} + r_k q_i - c_{ik} = 0$,
for all interior locations $i$ and tracers $k$, with $f_{ik} = c_{ik} - c_{bk}$
at the surface.  Mass conservation is one of these constraints, here
denoted to be the $k=K+1$ tracer: $f_{i,K+1} = \sum^N_{j=1} m_{ij} -
1$, found by substituting $c_{j,K+1}=1$, $c_{i,K+1}=1$, and $r_{K+1} =
0$ above.  Any steady-state tracer, $k$, must satisfy ${\cal F}_k = 0$
which is defined by appending $f_{ik}$ at all $i$ locations.

\subsection{Solution technique}
\label{sec:solution}

All constraints are enforced through the method of Lagrange
multipliers.  For equations that contain noise, a weighted quadratic
form, ${\bf n}^T {\bf W} {\bf n}$, is added to the Lagrangian function
in order to minimize the sum of squared noise elements.  For the
steady-state constraint that does not contain noise, a Lagrange
multiplier term is appended to the function for strict
enforcement. Solving for ${\bf n}$ in terms of the other unknowns of
the problem, the Lagrangian function is
\begin{eqnarray}
\label{eq:lagrangian}
   {\cal L} [{\bf c}_k,{\bf q},{\bf m}] &=&  (\sum^K_{k=1} \{ {\cal
     E}_k [{\bf c}_k] \} - {\bf
  y})^T ~{\bf W} ( \sum^K_{k=1} \{ {\cal E}_k [{\bf c}_k] \} - {\bf y}) \\
%\sum^K_{k=1}[\alpha_k ({\bf E c}_k-{\bf c}_k^{obs})^T 
%{\bf W}_k^{-1} ({\bf  E c}_k-{\bf c}_k^{obs})
&+&  \sum^K_{k=1} \{ (\Gamma {\bf c}_k - {\bf c}_{0k})^T {\bf
   S}_k   (\Gamma {\bf c}_k - {\bf c}_{0k}) \}\\
% & +&  \alpha_g {\cal F}_g [{\bf c}]^T {\bf W}_g^{-1}  {\cal
%    F}_g[ {\bf c}]    
% +      \alpha_{nn} {\cal F}_{nn} [{\bf c}]^T {\bf W}_{nn}^{-1}  {\cal
%    F}_{nn}[ {\bf c}]   
% +      \alpha_{as} ({\bf F}_{as} {\bf c})^T {\bf W}_{as}^{-1} ({\bf
%    F}_{as}{\bf c})    \\
& +&  {\cal F}_c [{\bf c}_k]^T ~{\bf W}_c  {\cal
   F}_c[ {\bf c}_k]  
 +  {\cal F}_q[{\bf q}]^T ~{\bf S}_q  {\cal F}_q
 [{\bf q}] +  {\cal F}_m [{\bf m}]^T ~{\bf S}_m {\cal
   F}_m [{\bf m}] \\ 
&+&\sum^{K+1}_{k=1}  \mu_k^T ~{\cal F}_k [{\bf c}_k,{\bf q},{\bf m}],
\label{eq:lagrangian2}
\end{eqnarray}
where ${\bf W}$ is the observational weighting matrix and ${\bf W}_c$ is
the weighting of the additional tracer constraints. ${\bf S}_k$, ${\bf
  S}_q$ and ${\bf S}_m$ are scaling matrices that enforce bounds and
smoothness on the expected deviation in each surface tracer, interior
source, and pathways, respectively. 
There is one Lagrange multiplier
vector, $\mu_k$, for each of the $K+1$ steady-state equations, and
note that the Lagrange multiplier terms do not alter the numerical
value of the function.

For tracer climatologies, ${\bf W}$ is diagonal with the inverse of published error estimates squared. For the sediment core data, the ${\bf W}$ matrix has the same form, and the additional constraint that there be no systematic misfit with depth is appended. Finding the model-data bias as a function of depth is a linear operation, ${\bf Z}{\bf n}$, an additional constraint can be added to the previous cost function term. The weighting matrix then becomes ${\bf W} + {\bf Z}^T {\bf W}_z {\bf Z}$, a nondiagonal matrix. ${\bf W}_z$ is chosen based on the expected vertical systematic error that arises from random noise.


The minimum of ${\cal L}$ is found by seeking a stationary point of
the terms ~(\ref{eq:lagrangian})-(\ref{eq:lagrangian2}). The partial
derivative with respect to each tracer gives a set of (adjoint)
equations that is solved for the Lagrange multipliers by an LU
decomposition.  The Lagrange multiplier vectors yield information
about how the Lagrangian function will change given a change in a
subset the unknowns, ${\bf c}_{bk}$, ${\bf q}$, and ${\bf m}$, where
these variables contain all information necessary to solve for the
global tracer distributions.  A quasi-Newton gradient descent method
uses this information to iteratively search for the minimum ${\cal L}$
\citep{Nocedal-1980:Updating,Gilbert-Lemarechal-1989:some}. 
All controls (independent unknowns) are preconditioned by the Cholesky
decomposition of the respective scaling matrix, ${\bf S}$, to improve
performance.

\section{Model-Data Misfits}

The use of nondiagonal matrices, as described in the method above, puts emphasis on capturing the vertical structure of the observations. The model-data misfit as a function of depth is included in Auxiliary Figures 1-2. Additionally, we include a meridional section of the $\delta^{13}$C and $\delta^{13}$C$_{as}$ distributions with pointwise misfits denoted  (Auxiliary Figures 3-4). These figures indicate that the vertical structure of the observations is captured within the expected errors.   

\begin{figure}
\includegraphics[width=40pc]{suppfigures/mod_errprofile_14may2013.png}
      \caption{Comparison of reconstructed error relative to expected
        error as a function of depth: standard deviation of
        model-observation misfit ({\it solid lines}) over a 1000 meter
        running interval for $\delta^{13}$C$_{\mbox{\tiny{DIC}}}$ ({\it
          left panel}), PO$_4$ ({\it middle panel}), and
        $\delta^{18}$O$_w$ ({\it right panel}), and the expected standard
        error from published sources or subjective choice as used in
        the reconstruction method ({\it dashed lines}).}
\end{figure}

\begin{figure}
\includegraphics[width=40pc]{suppfigures/lgm_errprofile_14may2013.png}
\caption{Error statistics of the LGM solution for $\delta^{13}$C ({\it left panel}), Cd ({\it middle panel}), and $\delta^{18}$O ({\it right panel}) data from benthic foraminiferal calcite. The reconstructed standard error over 1000 meter running depth intervals ($\sigma$, {\it bold solid line}) is compared to the expected standard error ({\it bold dashed line}). The absolute value of the mean misfit or offset ($\mu$) is compared between the reconstruction ({\it thin solid line}) and the expected 95\% confidence interval ({\it thin dashed line}).}
\end{figure}

\begin{figure}
\includegraphics[width=40pc]{suppfigures/nc13_lgm_watl_21aug2013.eps}
     \caption{The glacial meridional sections collapse all data points onto a western or eastern section. To account for zonal variations in the core locations, we plot the model-data misfit at the actual locations of the data. The model-data misfit is included for LGM $\delta^{13}$C$_c$. Squares represent data locations in the western Atlantic (west of $35^{\circ}$W) and circles represent the eastern Atlantic.}
\end{figure}

\begin{figure}
     \includegraphics[width=40pc]{suppfigures/lgm_watl_nc13as_21aug2013.eps}
     \caption{Same as the previous figure but for $\delta^{13}$C$_{as}$ inferred from core locations with both $\delta^{13}$C$_c$ and Cd measurements.}
\end{figure}


\section{Water-mass analysis}

A phosphate-$\delta^{13}$C diagram puts the geographic Atlantic
section into the context of water-mass endmember values and mixing
lines.  At the core sites that contain measurements of both Cd and
$\delta^{13}$C, we translate Cd to phosphate following
\citet{Elderfield-Rickaby-2000:Oceanic} and plot $\delta^{13}$C-PO$_4$
property combinations, where $\delta^{13}$C$_{as}$ isolines are
straight in this property-property space
(Auxiliary Figure~\ref{fig:watermasses}). Even though the $\delta^{13}$C$_{as}$
map of this work appears to have a different vertical structure from
the map of \citet{Marchitto-Broecker-2006:Deep}, the dominant linear
relationship between $\delta^{13}$C and phosphate is captured at the
core sites, including the modern-day slope of -0.6\permil/$\mu$mol/kg
steepening to -1.3\permil/$\mu$mol/kg during the LGM. All modern-day
$\delta^{13}$C$_{as}$ core values and all but three LGM
$\delta^{13}$C$_{as}$ core values are within 0.5\permil~ of zero after
accounting for observational noise, an indication that air-sea
disequilbrium signatures are small in the deep ocean if the recent
burning of fossil fuels is eliminated
\citep[e.g.,][]{Olsen-Ninnemann-2010:Large} and that the water-mass
signal in $\delta^{13}$C$_{as}$ is smaller than the signal of
biological effects.


\begin{figure}
    \label{fig:watermasses}
     \includegraphics[width=40pc]{figures/modvlgm_c13vspo4_21may2013.png}
     \caption{Water mass diagrams in phosphate-$\delta^{13}$C space
       for the modern ({\it left panel}) and LGM ({\it right panel}).
       Reconstructed tracer-tracer values at the locations with Cd and $\delta^{13}$C observations ({\it black circles}) are
       placed into the context of $\delta^{13}$C$_{as}$ ({\it
         background contours}). The effective endmember values ({\it
         open circles}) enclose
       the data points, and are defined for the following surface
       regions: Weddell Sea (WED), Atlantic Subantarctic (SUBANT),
       Labrador and Irminger Seas (LAB), Greenland-Icelandic-Norwegian
     Seas (GIN), Arctic (ARC), Mediterranean (MED), and
     subtropics/tropics (TROP). The modern-day MED endmember is offscale
     (0.2$\mu$mol/kg, 2.2\permil).}
      \label{fig:watermasses} 
\end{figure}


The change in phosphate-$\delta^{13}$C slope is due to the
``effective endmembers,'' defined to be the most representative
properties in a given surface region
\citep{Gebbie-Huybers-2011:How-is-the-ocean}.  Here we define seven
surface regions in the Atlantic sector (see
Figure~\ref{fig:watermasses} for definitions), where the endmembers
are computed by weighting the surface tracer concentrations by the
Atlantic interior volume filled by each location. The modern-LGM
difference in slope is due to the change in $\delta^{13}$C$_{as}$ of
the primary water-masses that make the general linear trend.
%: Weddell
%Sea Water and some combination of Labrador Sea and
%Greenland-Iceland-Norwegian (GIN) Sea water. 
In the modern-day, the Labrador and GIN Seas have relatively-high
$\delta^{13}$C but low $\delta^{13}$C$_{as}$ (-0.4\permil). Weddell
Sea Water has low $\delta^{13}$C but high $\delta^{13}$C$_{as}$
(0.3\permil), with the net effect being a flattening of the overall
slope. In the LGM, the pattern is reversed, with relatively-low
$\delta^{13}$C and $\delta^{13}$C$_{as}$ being colocated in the
Southern Ocean, and both quantities relatively high in the North
Atlantic. The net result is a steepened slope in the LGM. 

%Thus, the
%$\delta^{13}$C$_{as}$ deviations that reconcile $\delta^{13}$C and
%phosphate aren't very large, but they
% systematically occur in the
%extreme property values and thus can resolve the $\delta^{13}$C-Cd
%discrepancy previously noted \citep{Boyle-Rosenthal-1996:Chemical}.

% endmember estimates are -0.5\permil~ and -0.4\permil,
% respectively, while the Weddell Sea is 0.3\permil.  The glacial
% Weddell Sea Water, in contrast, has $\delta^{13}$C$_{as}$ of
% -0.7\permil, with the northern endmembers centered closer to zero. 
% %Furthermore, the change in slope is related to the vertical structure
% %of the tracers, and 
% In the South Atlantic, the LGM vertical Cd range of 0.1 nmol/kg to 0.7
% nmol/kg at $30^\circ$S translates to a 1.6 $\mu$mol/kg range in
% phosphate \citep{Elderfield-Rickaby-2000:Oceanic}. If remineralization
% is the dominant process, the range of $\delta^{13}$C would be expected
% to be 1.7\permil~ (1/0.95 times the phosphate range). Here a
% $\delta^{13}$C gradient of 2.0\permil~ is modeled and $\delta^{13}$C
% is lower than would be predicted from Cd and remineralization
% \citep{Boyle-Rosenthal-1996:Chemical}.  If the Antarctic
% $\delta^{13}$C$_{as}$ endmember can be as low as the -0.7\permil~
% reconstructed here, then there is no conundrum in the Southern Ocean
% $\delta^{13}$C-Cd relationship because the offset is only
% 0.3\permil. 

% I HAD TO KILL THIS PARAGRAPH DUE TO PALEO'S RESTRICTIONS.
 Even though the effective endmember values aren't strongly constrained
 by observations, there is general agreement between our study and
 previous studies. Glacial endmember estimates were previously
 estimated by extrapolation of mixing lines in a property-property
 diagram \citep[AABW: 0.6 to 0.8 nmol/kg, -0.5 to -1.0\permil~
 $\delta^{13}$C,][]{Marchitto-Broecker-2006:Deep}. Here, we refine
 their estimates (0.85 nmol/kg, -0.9\permil), but recognize that any
 additional Southern Ocean observations could alter them.  Glacial
 Antarctic Intermediate Water $\delta^{13}$C is reconstructed to be
 0.2-0.3\permil, in accord with \citet{Curry-Oppo-2005:glacial}. We
 also find a large gradient between upper and lower North Atlantic
 Water properties, consistent with increased sea ice and unutilized
 nutrients in the Nordic Seas \citep{Mix-Fairbanks-1985:North}, that
 further complicates the finding of an interface between northern and
 southern source waters, as lower North Atlantic Water is more similar
 to Antarctic water.  Other endmember values will need to be refuted or
 refined with additional observations. For example, the Atlantic-wide
 range of Cd is 50\% greater during the LGM, which together with the
 conservation of global phosphate inventory, requires North Atlantic
 source waters to be even more depleted than they are today (<0.1
 nmol/kg).

%\section{Western Atlantic Property Atlas}

% Additional meridional sections that compare modern and LGM properties along the western Atlantic GEOSECS track are included here for the following properties: Cd$_w$ and $\delta^{18}$O$_c$. Similar figures for $\delta^{13}$C, $\delta^{13}$C$_{as}$, preformed $\delta^{13}$C, and North Atlantic Water concentration, ${\bf g}_{north}$, were included in the main text.

\begin{figure}
     \includegraphics[width=40pc]{suppfigures/modvlgm_watl_cd_8jan2014.png}
     \caption{{\bf Western Atlantic Property Atlas}: Additional meridional sections that compare modern and LGM properties along the western Atlantic GEOSECS track are included here for the following properties: Cd$_w$ (this figure) and $\delta^{18}$O$_c$ (next figure). Similar figures for $\delta^{13}$C, $\delta^{13}$C$_{as}$, preformed $\delta^{13}$C, and North Atlantic Water concentration, ${\bf g}_{north}$, are included in the main text.}
\end{figure}

\begin{figure}
     \includegraphics[width=40pc]{suppfigures/modvlgm_watl_o18c_8jan2014.png}
     \caption{Same as previous figure but for $\delta^{18}$O$_c$.}
\end{figure}

%\section{Eastern Atlantic Property Atlas}

% Eastern Atlantic meridional sections along the A16 WOCE transect are included here for the following properties:  $\delta^{13}$C, $\delta^{13}$C$_{as}$, Cd$_w$, $\delta^{18}$O$_c$, and North Atlantic Water concentration, ${\bf g}_{north}$. 

\begin{figure}
     \includegraphics[width=40pc]{suppfigures/modvlgm_eatl_c13_8jan2014.png}
     \caption{{\bf Eastern Atlantic Property Atlas}: Meridional sections along the A16 WOCE transect are included here for the following properties:  $\delta^{13}$C, $\delta^{13}$C$_{as}$, Cd$_w$, $\delta^{18}$O$_c$, and North Atlantic Water concentration, ${\bf g}_{north}$.}
\end{figure}

\begin{figure}
     \includegraphics[width=40pc]{suppfigures/modvlgm_eatl_c13as_8jan2014.png}
     \caption{Same as Figure 8, but for $\delta^{13}$C$_{as}$.}
\end{figure}

\begin{figure}
     \includegraphics[width=40pc]{suppfigures/modvlgm_eatl_cd_8jan2014.png}
     \caption{Same as Figure 8, but for seawater Cd.}
\end{figure}

\begin{figure}
     \includegraphics[width=40pc]{suppfigures/modvlgm_eatl_o18c_8jan2014.png}
     \caption{Same as Figure 8, but for $\delta^{18}$O$_{c}$.}
\end{figure}

\begin{figure}
     \includegraphics[width=40pc]{suppfigures/modvlgm_eatl_gnorth_8jan2014.png}
     \caption{Same as Figure 8, but for North Atlantic Water concentration, ${\bf g}_{north}$.}
\end{figure}

%\section{Model-Data Misfits}

% The glacial meridional sections collapse all data points onto a western or eastern section. To account for zonal variations in the core locations, we plot the model-data misfit at the actual locations of the data. The model-data misfit is included for LGM $\delta^{13}$C$_c$, inferred seawater Cd at the core locations, and $\delta^{18}$O$_c$.
%% VERTICAL STRUCTURE OF ERRORS.  


%\acknowledgment{We thank an anonymous reviewer for simplifying the
%  derivation of the adjoint calculation for surface origins.} 

\bibliographystyle{agu08}
\bibliography{all}

\end{document}




