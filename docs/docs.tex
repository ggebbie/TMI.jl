
\documentclass[agums]{aguplus}  % use this variant for AGU manuscripts
%\documentclass[agums]{aguplus_ams}  % use this variant for to get parentheses for 
                                    % American Meteorlogical Society publications
%\usepackage{psfig,times}
\usepackage{wasysym}
\usepackage{graphicx}
\usepackage{pslatex}
\usepackage{subfigure}
% \usepackage{setspace}
 \usepackage[dvips]{epsfig}
 \setlength{\oddsidemargin}{0.1cm}
 \setlength{\evensidemargin}{0.1cm}
 \setlength{\textwidth}{16cm}
 \setlength{\textheight}{22.1cm}
 \setlength{\topmargin}{.0cm} 
%\usepackage{amsmath}
\usepackage[latin1]{inputenc}
\def\degree{\hbox{$^\circ$}}
\usepackage[T1]{fontenc}
\def\scalefig#1{\epsfxsize #1\textwidth}

\makeatletter
%\def\@sluginfo{{\vspace{1in}in preparation, \today}}
\renewcommand\revtex@pageid{}
\makeatother


%\renewcommand\NAT@open{(} \renewcommand\NAT@close{)}
%\makeatother
\sectionnumbers
%\printfigures
%\tighten  %uncomment this line for single-spaced text
\lefthead{Gebbie}

\righthead{Auxiliary Text}
\received{}
\revised{}
\accepted{}
\journalid{}{}
\articleid{}{}
\paperid{}
\cpright{}{}
\ccc{}
%\printfigures


\begin{document}

% \title{Auxiliary Text and Figures: How is the ocean filled?}

% \author{Geoffrey Gebbie$^{1,2,3}$ and Peter Huybers$^1$}

\title{}
% \affil{$^1$ Harvard University, Cambridge, MA, $^2$ Massachusetts Institute of Technology, Cambridge, MA, $^3$ Now at Woods Hole Oceanographic Institution, Woods Hole, MA} 
       

% %\author{In press, {\it Geophys. Geochem. Geosyst.}, \today}  % place holder for drafts
% \authoraddr{Geoffrey Gebbie, Woods Hole Oceanographic Institution, MS \# 29, 266 Woods Hole Rd., Woods Hole, MA, 02543, USA.}

% %% agu abstracts. GRL.
% %   *  Set as a single paragraph
% %    * Limit to 250 words for journals
% %    * Do not include references in the abstract

%\begin{abstract}

% \end{abstract}

% \begin{flushright}
% \textbf{Auxiliary Information}
% \end{flushright}


% \vspace{1in}

% \begin{center}
% \large{\textbf{How is the ocean filled?}}\\
% \vspace{.25in}

% G. Gebbie \verb+(ggebbie@whoi.edu)+ and P. Huybers.

% \end{center}


% \vspace{.5in}

% \begin{itemize}

% \item Auxiliary Notes and Figures

% \begin{itemize}

% \item       The steady-state circulation

% \item	Estimating tracer distributions

% \item	Diagnosing water-mass pathways

% \item Adjoint method for the surface origination map

% \item The distribution of water masses and its sensitivity to spatial resolution

% \end{itemize}

% \item References

% \item Figures

% \end{itemize}

%\Large{\textbf{Auxiliary Text and Figures}} \\

%\normalsize

\begin{itemize}
 \item[] {\bf Auxiliary Material Submission for manuscript} \\
  ``How much did Glacial North Atlantic Water shoal?'' \\ \\
Geoffrey Gebbie \\
Physical Oceanography Department, Woods
Hole Oceanographic Institution \\ \\

\item[]    Contents 

\item[1.] Detailed inverse problem formulation

\item[2.] Model-Data Misfits

\item[3.] Water-mass analysis

\item[4.] References

\item[5.] Figures

\item[5b.] Western Atlantic Property Atlas

\item[5c.] Eastern Atlantic Property Atlas


%\item[3.] Brazil Margin analysis


%\item Gebbie, G., and P. Huybers, How is the ocean filled?, 2011. 

 \end{itemize}

% Introduction 


% 1. 2010jd014771-txts01.doc: Text S1: 
% (1) A summary of the infrared spectra for Figures 1 and 5 in the overall range that were measured, along with the theoretical ones in the 0-2500 cm-1 wavenumber range. 
% (2) The final structures used in theoretical calculation in Cartesian coordinate format for all compound from Tables 3 and 5. Calculation were run in Gaussian03 with the B3LYP/6-31G** level of theory. X, Y and Z represent the coordinates in A (deg).

\newpage

\section{Detailed inverse problem formulation}
\label{sec:proxy}

The method is detailed through a set of constraints on the problem
unknowns: the water-mass proportions, ${\bf m}$, all relevant tracer
distributions on an underlying grid, ${\bf c}_k$, and any
remineralization source for nonconservative tracers, ${\bf q}$.

%\subsection{Constraints}

Tracer observations are imposed without loss of generality as an
equation: $\sum^K_{k=1} \{ {\cal E}_k[ {\bf c}_k] \} = {\bf y} + {\bf
  n}$, where there are $K$ modeled tracers, ${\cal E}_k$ is a
potentially nonlinear mapping of the gridded fields onto the observational
location of observational type and performs the ``proxy'' step of
relating the modeled tracers to observed quantities, ${\bf y}$ is the
list of many disparate observations, and ${\bf n}$ is the noise in the
observations. This form takes into account the fact that
paleo-observations may depend upon multiple seawater tracers (e.g.,
$\delta^{18}$O$_c$ requires temperature and seawater $\delta^{18}$O),
and that this relationship may be nonlinear (e.g., $\delta^{18}$O$_c$
requires in-situ rather than potential temperature, and Cd$_w$ is
nonlinearly related to phosphate). If the tracer observations are
restricted to those that are explicitly modeled (e.g.,
$\delta^{18}$O$_w$) or to climatologies where a tracer value is
available at all gridded locations (e.g., WOCE salinity and
nutrients), the nonlinear function ${\cal E}_k$ is reduced to a
matrix. % or even drops out of the problem entirely.

The tracers are constrained by non-observational information, as well.
These constraints include gravitational stability, non-negativity
constraints expressed in terms of barrier functions, and
$\delta^{13}$C$_{as}$ being within the range that has been found in
modern-day observations and models ($\pm 1\permil$).  To keep
the solution within reasonable bounds and also improve the efficiency
of the solution method, the surface tracer deviation is defined
relative to a first guess, $\Gamma {\bf c}_{bk} = {\bf c}_{0k} + {\bf
  n}_{bk}$, where $\Gamma$ picks out the surface values from the
global distribution, ${\bf c}_{bk}$ is the surface concentration,
${\bf c}_{0k}$ is the first guess, and ${\bf n}_{bk}$ is surface
deviation away from the first guess for tracer $k$. Equations for
gravitational stability, non-negativity, and the range of
$\delta^{13}$C$_{as}$ are symbolically lumped into one set of
nonlinear equations for the derivation here: ${\cal F}_c[{\bf c}_k] =
{\bf n}_c$, where ${\bf n}_c$ is the degree to which these equations
are inexact.

No direct observations of the interior tracer sources, ${\bf q}$, or
the water-mass pathway vector, ${\bf m}$, are available, so these
unknowns are simply kept within a reasonable range of a first-guess
where, for example, ${\cal F}_q[{\bf q},{\bf q}_0]=\mbox{log}({\bf q})
- \mbox{log}({\bf q}_0)={\bf n}_q$, is the deviation from the first
guess local sources, ${\bf q}_0$. The difference of logarithms assumes
and enforces the source to be positive, and for the order of magnitude
of the deviation to be penalized. The pathway parameters, ${\bf m}$,
are penalized using an similar nonlinear function, ${\cal F}_m$.

\subsection{Steady state constraint}

Equation~(1) in the main text is put into the form of a steady-state
constraint by putting all terms on the left hand side:
$f_{ik} = \sum^N_{j=1} m_{ij}~ c_{jk} + r_k q_i - c_{ik} = 0$, for all
interior locations $i$ and tracers $k$, with
$f_{ik} = c_{ik} - c_{bk}$ at the surface.  Mass conservation is one
of these constraints, here denoted to be the $k=K+1$ tracer:
$f_{i,K+1} = \sum^N_{j=1} m_{ij} - 1$, found by substituting
$c_{j,K+1}=1$, $c_{i,K+1}=1$, and $r_{K+1} = 0$ above.  Any
steady-state tracer, $k$, must satisfy ${\cal F}_k = 0$ which is
defined by appending $f_{ik}$ at all $i$ locations.

\subsection{Solution technique}
\label{sec:solution}

All constraints are enforced through the method of Lagrange
multipliers.  For equations that contain noise, a weighted quadratic
form, ${\bf n}^T {\bf W} {\bf n}$, is added to the Lagrangian function
in order to minimize the sum of squared noise elements.  For the
steady-state constraint that does not contain noise, a Lagrange
multiplier term is appended to the function for strict
enforcement. Solving for ${\bf n}$ in terms of the other unknowns of
the problem, the Lagrangian function is
\begin{eqnarray}
\label{eq:lagrangian}
   {\cal L} [{\bf c}_k,{\bf q},{\bf m}] &=&  (\sum^K_{k=1} \{ {\cal
     E}_k [{\bf c}_k] \} - {\bf
  y})^T ~{\bf W} ( \sum^K_{k=1} \{ {\cal E}_k [{\bf c}_k] \} - {\bf y}) \\
%\sum^K_{k=1}[\alpha_k ({\bf E c}_k-{\bf c}_k^{obs})^T 
%{\bf W}_k^{-1} ({\bf  E c}_k-{\bf c}_k^{obs})
&+&  \sum^K_{k=1} \{ (\Gamma {\bf c}_k - {\bf c}_{0k})^T {\bf
   S}_k   (\Gamma {\bf c}_k - {\bf c}_{0k}) \}\\
% & +&  \alpha_g {\cal F}_g [{\bf c}]^T {\bf W}_g^{-1}  {\cal
%    F}_g[ {\bf c}]    
% +      \alpha_{nn} {\cal F}_{nn} [{\bf c}]^T {\bf W}_{nn}^{-1}  {\cal
%    F}_{nn}[ {\bf c}]   
% +      \alpha_{as} ({\bf F}_{as} {\bf c})^T {\bf W}_{as}^{-1} ({\bf
%    F}_{as}{\bf c})    \\
& +&  {\cal F}_c [{\bf c}_k]^T ~{\bf W}_c  {\cal
   F}_c[ {\bf c}_k]  
 +  {\cal F}_q[{\bf q}]^T ~{\bf S}_q  {\cal F}_q
 [{\bf q}] +  {\cal F}_m [{\bf m}]^T ~{\bf S}_m {\cal
   F}_m [{\bf m}] \\ 
&+&\sum^{K+1}_{k=1}  \mu_k^T ~{\cal F}_k [{\bf c}_k,{\bf q},{\bf m}],
\label{eq:lagrangian2}
\end{eqnarray}
where ${\bf W}$ is the observational weighting matrix and ${\bf W}_c$
is the weighting of the additional tracer constraints. ${\bf S}_k$,
${\bf S}_q$ and ${\bf S}_m$ are scaling matrices that enforce bounds
and smoothness on the expected deviation in each surface tracer,
interior source, and pathways, respectively.  There is one Lagrange
multiplier vector, $\mu_k$, for each of the $K+1$ steady-state
equations, and note that the Lagrange multiplier terms do not alter
the numerical value of the function.

For tracer climatologies, ${\bf W}$ is diagonal with the inverse of
published error estimates squared. For the sediment core data, the
${\bf W}$ matrix has the same form, and the additional constraint that
there be no systematic misfit with depth is appended. Finding the
model-data bias as a function of depth is a linear operation,
${\bf Z}{\bf n}$, an additional constraint can be added to the
previous cost function term. The weighting matrix then becomes
${\bf W} + {\bf Z}^T {\bf W}_z {\bf Z}$, a nondiagonal matrix.
${\bf W}_z$ is chosen based on the expected vertical systematic error
that arises from random noise.


The minimum of ${\cal L}$ is found by seeking a stationary point of
the terms ~(\ref{eq:lagrangian})-(\ref{eq:lagrangian2}). The partial
derivative with respect to each tracer gives a set of (adjoint)
equations that is solved for the Lagrange multipliers by an LU
decomposition.  The Lagrange multiplier vectors yield information
about how the Lagrangian function will change given a change in a
subset the unknowns, ${\bf c}_{bk}$, ${\bf q}$, and ${\bf m}$, where
these variables contain all information necessary to solve for the
global tracer distributions.  A quasi-Newton gradient descent method
uses this information to iteratively search for the minimum ${\cal L}$
\citep{Nocedal-1980:Updating,Gilbert-Lemarechal-1989:some}.  All
controls (independent unknowns) are preconditioned by the Cholesky
decomposition of the respective scaling matrix, ${\bf S}$, to improve
performance.

\bibliographystyle{agu08}
\bibliography{all}

\end{document}




